\chapter{Propuestas, Generalidades y Configuraciones Utilizadas}

\section{Motivación del Estudio}

La creciente necesidad de tratamientos multidrogas para abordar enfermedades complejas ha impulsado el interés en identificar interacciones entre medicamentos (\textit{Drug-Drug Interactions}, DDIs). Estas interacciones pueden ocasionar efectos secundarios adversos o incluso consecuencias graves para la salud. La detección experimental de DDIs es un proceso costoso y prolongado, lo que subraya la importancia de los enfoques computacionales para predecir estas interacciones de manera más eficiente y precisa.

El artículo \textit{DPDDI: A Deep Predictor for Drug-Drug Interactions} presenta un método basado en técnicas de aprendizaje profundo, como las redes de convolución de grafos (\textit{Graph Convolutional Networks}, GCN) y las redes neuronales profundas (\textit{Deep Neural Networks}, DNN). El enfoque DPDDI utiliza información topológica de las redes de interacción de medicamentos para predecir DDIs con alta precisión.

\section{Propuesta del Artículo}

El enfoque DPDDI propuesto combina un extractor de características basado en GCN con un predictor basado en DNN. Las principales contribuciones de este método incluyen:

\begin{itemize}
    \item \textbf{Extracción de características latentes:} El uso de GCN permite capturar las relaciones topológicas entre los nodos (medicamentos) en una red de DDIs, generando representaciones de baja dimensionalidad que preservan información estructural clave.
    \item \textbf{Agregación de características:} Las características de pares de medicamentos se combinan mediante una operación de concatenación, que demostró ser superior a otros métodos como el producto interno y la suma.
    \item \textbf{Predicción con redes neuronales profundas:} El predictor basado en DNN aprende relaciones no lineales entre las características de los medicamentos para determinar la probabilidad de interacción.
\end{itemize}

El artículo reporta que DPDDI supera a cuatro métodos de última generación en diversas métricas, incluyendo el área bajo la curva ROC (AUC), la precisión y el \textit{F1-score}, demostrando su efectividad y robustez.

\section{Generalidades de los Ejercicios}

Para validar el modelo DPDDI, los autores utilizaron tres conjuntos de datos provenientes de \textit{DrugBank}:

\begin{itemize}
    \item \textbf{DB1:} Contiene 1562 medicamentos y 180,576 interacciones anotadas.
    \item \textbf{DB2:} Un subconjunto más pequeño con 548 medicamentos y 48,584 interacciones.
    \item \textbf{DB3:} Un conjunto más grande con 1934 medicamentos y 230,887 interacciones.
\end{itemize}

Los experimentos se realizaron utilizando validación cruzada de 5 particiones (\textit{5-fold cross-validation}) para evaluar el rendimiento del modelo en diferentes configuraciones.

\section{Configuraciones Utilizadas}

El artículo describe un extenso proceso de ajuste de hiperparámetros para optimizar tanto el extractor de características GCN como el predictor DNN. Las configuraciones óptimas utilizadas son las siguientes:

\subsection{Extractor de Características (GCN)}
\begin{itemize}
    \item \textbf{Tasa de aprendizaje:} 0.001
    \item \textbf{Épocas:} 1400
    \item \textbf{Tasa de abandono:} 0.0001
    \item \textbf{Dimensiones de la capa oculta:} [512, 128]
\end{itemize}

\subsection{Predictor (DNN)}
\begin{itemize}
    \item \textbf{Tasa de aprendizaje:} 0.01
    \item \textbf{Épocas:} 140
    \item \textbf{Tasa de abandono:} 0.001
    \item \textbf{Tamaño de lote:} 50
    \item \textbf{Dimensiones de la capa oculta:} [128, 64, 32]
\end{itemize}

\textbf{Nota: Esta configuración es la utilizada en el artículo, durante la replicación de los experimentos no se pudieron utilizar más de 30 epocas}


\section{Reproducción de los Ejercicios}

En el presente trabajo, los ejercicios descritos en el artículo original se reprodujeron utilizando la plataforma \textit{Therapeutic Data Commons}. Este proceso incluyó:

\begin{itemize}
    \item Implementación del modelo GCN para extraer características latentes de los medicamentos.
    \item Agregación de características de pares de medicamentos mediante concatenación.
    \item Entrenamiento de un predictor DNN con los mismos hiperparámetros reportados.
\end{itemize}

Los resultados obtenidos se compararon con los reportados en el artículo original para validar la reproducibilidad y efectividad del enfoque.
