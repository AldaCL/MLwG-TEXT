\chapter*{Introducción}
\addcontentsline{toc}{chapter}{Abstract}

En el ámbito de la medicina, el tratamiento de enfermedades complejas mediante combinaciones de medicamentos se ha vuelto una práctica cada vez más común. Sin embargo, las interacciones entre medicamentos (\textit{Drug-Drug Interactions}, DDIs) pueden generar efectos adversos inesperados e incluso toxicidad grave, representando un riesgo significativo para los pacientes. La detección de DDIs en entornos de laboratorio es un proceso costoso y que consume mucho tiempo. Por ello, se han desarrollado métodos computacionales como una alternativa eficiente para predecir estas interacciones.

En este trabajo, se reproduce la metodología planteada en el artículo \textit{DPDDI: A Deep Predictor for Drug-Drug Interactions}, el cual propone un enfoque novedoso basado en redes neuronales profundas (\textit{Deep Neural Networks}, DNN) y redes de convolución de grafos (\textit{Graph Convolutional Networks}, GCN). La metodología DPDDI utiliza las características estructurales de una red de DDIs para representar medicamentos en un espacio de características latentes de baja dimensionalidad. Posteriormente, dichas representaciones se integran para predecir interacciones potenciales entre medicamentos. Este enfoque demostró superar a otros métodos en términos de precisión y robustez, especialmente en escenarios donde no se dispone de propiedades químicas o biológicas detalladas de los medicamentos.

La reproducción de los experimentos del artículo original se realizó mediante el uso de la plataforma \textit{Therapeutic Data Commons}, enfocándose en evaluar la efectividad del modelo propuesto, así como explorar posibles extensiones en su implementación. Los resultados obtenidos resaltan la aplicabilidad de DPDDI en el diseño de tratamientos médicos más seguros y en la prevención de efectos secundarios inesperados. 